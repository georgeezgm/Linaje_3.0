% -*- mode: noweb; noweb-default-code-mode: R-mode; -*-
\documentclass[a4paper]{article}

\title{A Test File}
\author{Friedrich Leisch}


<<include=FALSE>>=
library(knitr)
opts_chunk$set(
echo=FALSE
)
@

\usepackage{a4wide}

\begin{document}

\maketitle

A simple example: the integers from 1 to 10 are
<<>>=
1:10
<<results='hide'>>=
print(1:20)
@ % the above is just to ensure that 2 code chunks can follow each other

We can also emulate a simple calculator:
<<echo=TRUE>>=
1 + 1
1 + pi
sin(pi/2)
@

Now we look at Gaussian data:

<<>>=
library(stats)
x <- rnorm(20)
print(x)
print(t1 <- t.test(x))
@
Note that we can easily integrate some numbers into standard text: The
third element of vector \texttt{x} is \Sexpr{x[3]}, the
$p$-value of the test is \Sexpr{format.pval(t1$p.value)}. % $

Now we look at a summary of the famous \texttt{iris} data set, and we
want to see the commands in the code chunks:


<<include=FALSE>>=
library(knitr)
opts_chunk$set(
echo=TRUE
)
@


<<>>=
data(iris)
summary(iris)
@ %def


\begin{figure}[htbp]
  \begin{center}
<<>>=
library(graphics)
pairs(iris)
@
     \caption{Pairs plot of the iris data.}
  \end{center}
\end{figure}

\begin{figure}[htbp]
  \begin{center}
<<>>=
boxplot(Sepal.Length~Species, data=iris)
@
    \caption{Boxplot of sepal length grouped by species.}
  \end{center}
\end{figure}

\end{document}
